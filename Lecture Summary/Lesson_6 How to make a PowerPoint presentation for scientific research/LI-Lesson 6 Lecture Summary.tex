\documentclass[prodmode,acmtap]{acmlarge}

% Metadata Information
\acmVolume{2}
\acmNumber{3}
\acmArticle{1}
\articleSeq{1}
\acmYear{2010}
\acmMonth{5}

% Package to generate and customize Algorithm as per ACM style
\usepackage[ruled]{algorithm2e}
\SetAlFnt{\algofont}
\SetAlCapFnt{\algofont}
\SetAlCapNameFnt{\algofont}
\SetAlCapHSkip{0pt}
\IncMargin{-\parindent}
\renewcommand{\algorithmcfname}{ALGORITHM}

% Page heads
\markboth{D. Pineo, C. Ware and S. Fogarty}{Neural Modeling of Flow Rendering Effectiveness}

% Title portion
\title{Postgraduate Research Enhancement Program: How to make a PowerPoint presentation for scientific research.}
\author{WeijieLi \affil{South China Normal University}}


\begin{abstract}



\end{abstract}



\begin{document}

\maketitle

The topic of Professor Zhou Chengju’s lecture is how to make a PowerPoint presentation for scientific research. 

First of all, Professor Zhou hoped that the students would understand the purpose of making a scientific research report PPT. At the graduate level, students will have many opportunities to participate in scientific research projects. Usually, you need to demonstrate your results through a defense, then you will need to use PPT. In addition, some papers also need to use PPT to introduce their papers during the submission process. 

Next, Professor Zhou introduced what should be included in the scientific research report PPT. The research background and objectives of the project should first be presented in the PPT. Start the method introduction on this basis. The experimental results and analysis obtained using this method are also key points. Towards the end, you need to summarize the conclusions and plan future work. During the whole process, Professor Zhou explained by using a PPT he made during his doctor study as an example.

Finally, Professor Zhou reminded the students about the format planning and style of the PPT for scientific research reports. In summary, each page of PPT needs to clearly express the core content. Let the reader be able to efficiently capture the meaning you want to convey. Must uphold rigorous principles. 

Professor Zhou Chengju’s lecture has benefited the students a lot, thank you very much. 


\end{document}
% End of v2-acmlarge-sample.tex (March 2012) - Gerry Murray, ACM

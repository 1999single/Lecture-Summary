\documentclass[prodmode,acmtap]{acmlarge}

% Metadata Information
\acmVolume{2}
\acmNumber{3}
\acmArticle{1}
\articleSeq{1}
\acmYear{2010}
\acmMonth{5}

% Package to generate and customize Algorithm as per ACM style
\usepackage[ruled]{algorithm2e}
\SetAlFnt{\algofont}
\SetAlCapFnt{\algofont}
\SetAlCapNameFnt{\algofont}
\SetAlCapHSkip{0pt}
\IncMargin{-\parindent}
\renewcommand{\algorithmcfname}{ALGORITHM}

% Page heads
\markboth{D. Pineo, C. Ware and S. Fogarty}{Neural Modeling of Flow Rendering Effectiveness}

% Title portion
\title{Postgraduate Research Enhancement Program: How to write a scientific research paper.}
\author{WeijieLi \affil{South China Normal University}}


\begin{abstract}



\end{abstract}



\begin{document}

\maketitle

The topic of Professor Qiu Lina's lecture is how to write a scientific research paper.

Before discussing this topic, it is very necessary to clarify the eligibility criteria of the paper. Professor Qiu introduced the review process of scientific research papers to the students. It is generally divided into three stages: preliminary review by the editorial department, peer review and processing by the editorial department. Professor Qiu listed 5 review tables, in which you can find the matters that the reviewers will be more concerned about.

Next, Professor Qu introduced the two editing environments of LATEX, is online editing and offline editing. Professor Qu mainly introduced an online editing website (overleaf). In addition to explaining how to use overleaf, the professor also explained the LATEX structure to the students. And how to find the template file and use the template.

Next, Professor Qiu introduced the main structure of scientific research papers using two types of paper catalogs as examples. In fact, the structure of the paper will be different. The professor recommends that the author guide prevail. There are some structures that most papers will include. Such as Authors and Affiliation, Abstract, Keywords, and Introduction, etc. Professor Qiu explained their functions, main content and noteworthy issues one by one around these common structures.

Finally, just mastering good writing skills is not enough. Professor Qiu emphasized that scientific integrity and ethics should be observed in the process of publishing papers. Reject false papers, multiple submissions per manuscript, and other issues that are easy to overlook.

Thank you very much for the guidance of Professor Qiu.

\end{document}
% End of v2-acmlarge-sample.tex (March 2012) - Gerry Murray, ACM

\documentclass[prodmode,acmtap]{acmlarge}

% Metadata Information
\acmVolume{2}
\acmNumber{3}
\acmArticle{1}
\articleSeq{1}
\acmYear{2010}
\acmMonth{5}

% Package to generate and customize Algorithm as per ACM style
\usepackage[ruled]{algorithm2e}
\SetAlFnt{\algofont}
\SetAlCapFnt{\algofont}
\SetAlCapNameFnt{\algofont}
\SetAlCapHSkip{0pt}
\IncMargin{-\parindent}
\renewcommand{\algorithmcfname}{ALGORITHM}

% Page heads
\markboth{D. Pineo, C. Ware and S. Fogarty}{Neural Modeling of Flow Rendering Effectiveness}

% Title portion
\title{Postgraduate Research Enhancement Program: How to use LATEX to write scientific papers.}
\author{WeijieLi \affil{South China Normal University}}


\begin{abstract}



\end{abstract}



\begin{document}

\maketitle

This lecture was presided over by Professor Qu Chao. The topic of the lecture is how to use LATEX to write scientific papers.

First, Professor Qu introduced the background and design concept of LATEX. Compared with traditional text editing software, LATEX has the advantage that it can edit formatted files, edit files for formatting, present content separation and be suitable for multi-person collaboration. Intuitively, researchers can directly use the template provided by the publisher and fill in the content. Save the time for the tedious operation of adjusting the format.

Next, Professor Qu introduced the two editing environments of LATEX, is online editing and offline editing. Professor Qu mainly introduced an online editing website (overleaf). In addition to explaining how to use overleaf, the professor also explained the LATEX structure to the students. And how to find the template file and use the template.

It is difficult to appreciate the superiority of LATEX just by explaining. Professor Qu began to demonstrate with mathematical formula editing as an example, showing students a variety of insertion methods. In addition, it also demonstrates the insert operation of the table. Obviously, performing this series of operations on LATEX is more flexible and free.

Finally, Professor Qu introduced how to use references. A specific type of file management reference (.bib) is usually used in LATEX. The separation of operations makes the management of references more targeted and more convenient. The professor recommends using jabref to manage this type of file (.bib).

Thank you very much Professor Qu for your guidance.

\end{document}
% End of v2-acmlarge-sample.tex (March 2012) - Gerry Murray, ACM

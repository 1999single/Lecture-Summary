\documentclass[UTF-8]{ctexart}
\CTEXsetup[format={\Large\bfseries}]{section}
\usepackage[bottom=3.5cm,top=3.5cm,left=3cm,right=3cm]{geometry}
\title{Postgraduate Research Enhancement Program: Postgraduate study, research and career planning}
\author{WeijieLi \quad 华南师范大学}
\date{2021.7.2}
\begin{document}
\maketitle
\thispagestyle{empty}
\clearpage

\pagestyle{plain}
\setcounter{page}{1}
\pagenumbering{arabic}

\section{Lecture Summary}
The topic of Professor Liang's lecture is graduate career planning. It is very necessary to hold this lecture before prospective graduate students step into the door of scientific research. First of all, the professor pointed out that the difference between postgraduates and undergraduates is that postgraduates should focus on the combination of research + innovation, rather than just stop at learning and application. Next, Professor Liang introduced the arrangements for each semester of graduate students, so that students have a simple understanding of their own learning route. The point is here. On how to conduct learning and research, Professor Liang gave a suggestion: read a lot of literature. There are different requirements in different periods. In the early stage, I have contact with review-type literature. The language is mainly in my mother tongue, which can help me adapt to the rhythm. At the stage of in-depth study, you should get in touch with some English literature, especially classic literature from authoritative journals. Of course, just staying at the reading stage is not enough. If you want to make a breakthrough, you must reproduce the literature to get a clearer understanding of the advantages and disadvantages of the literature. Professor Liang suggested that students set some goals to motivate themselves to move forward. At the same time, time management must be done well. Communicate more with tutors and classmates, and extensively study the basic knowledge of the subject. Finally, I would like to thank Professor Liang for his guidance. 
\end{document}